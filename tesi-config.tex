%**************************************************************
% file contenente le impostazioni della tesi
%**************************************************************

%**************************************************************
% Frontespizio
%**************************************************************
\newcommand{\myName}{Bogdan Ionut Suierica}                                    % autore
\newcommand{\myTitle}{Tres - Tree recommendation system}                    
\newcommand{\myDegree}{Tesi di laurea triennale}                % tipo di tesi
\newcommand{\myUni}{Università degli Studi di Padova}           % università
\newcommand{\myFaculty}{Corso di Laurea in Informatica}         % facoltà
\newcommand{\myDepartment}{Dipartimento di Matematica}          % dipartimento
\newcommand{\myProf}{Tullio Vardanega}                                % relatore
\newcommand{\myLocation}{Padova}                                % dove
\newcommand{\myAA}{2015-2016}                                   % anno accademico
\newcommand{\myTime}{Dec 2015}                                  % quando


%**************************************************************
% Impostazioni di impaginazione
% see: http://wwwcdf.pd.infn.it/AppuntiLinux/a2547.htm
%**************************************************************

\setlength{\parindent}{14pt}   % larghezza rientro della prima riga
\setlength{\parskip}{0pt}   % distanza tra i paragrafi


%**************************************************************
% Impostazioni di biblatex
%**************************************************************
\bibliography{bibliografia} % database di biblatex 

\defbibheading{bibliography}
{
    \cleardoublepage
    \phantomsection 
    \addcontentsline{toc}{chapter}{\bibname}
    \chapter*{\bibname\markboth{\bibname}{\bibname}}
}

\setlength\bibitemsep{1.5\itemsep} % spazio tra entry

\DeclareBibliographyCategory{opere}
\DeclareBibliographyCategory{web}

%\addtocategory{opere}{womak:lean-thinking}
\addtocategory{web}{site:agile-manifesto}

\defbibheading{opere}{\section*{Riferimenti bibliografici}}
\defbibheading{web}{\section*{Siti Web consultati}}


%**************************************************************
% Impostazioni di caption
%**************************************************************
\captionsetup{
    tableposition=top,
    figureposition=bottom,
    font=small,
    format=hang,
    labelfont=bf
}

%**************************************************************
% Impostazioni di glossaries
%**************************************************************

%**************************************************************
% Acronimi
%**************************************************************
\renewcommand{\acronymname}{Acronimi e abbreviazioni}

\newacronym[description={\glslink{apig}{Application Program Interface}}]
    {api}{API}{Application Program Interface}

\newacronym[description={\glslink{umlg}{Unified Modeling Language}}]
    {uml}{UML}{Unified Modeling Language}
    
\newacronym[description={\glslink{ictg}{Information and Communications Technology}}]
	{ict}{ICT}{Information and Communications Technology}
	
\newacronym[description={\glslink{cmsg}{Content Management System}}]
{cms}{CMS}{Content Management System}

\newacronym[description={\glslink{mvcg}{Model-View-Controller}}]
{mvc}{MVC}{Model-View-Controller}

\newacronym[description={\glslink{ideg}{Integrated Development Environment}}]
{IDE}{IDE}{Integrated Development Environment}

\newacronym[description={\glslink{restg}{Representational State Transfer}}]
{REST}{REST}{Representational State Transfer}

\newacronym[description={\glslink{urig}{Uniform Resource Identifier}}]
{URI}{URI}{Uniform Resource Identifier}

\newacronym[description={\glslink{jsong}{Javascript Object Notation}}]
{JSON}{JSON}{Javascript Object Notation}

\newacronym[description={\glslink{jvmg}{Java Virtual Machine}}]
{JVM}{JVM}{Java Virtual Machine}

\newacronym[description={\glslink{daog}{Data Access Object}}]
{dao}{DAO}{Data Access Object}

\newacronym[description={\glslink{pdca}{Plan-Do-Check-Act}}]
{pdca}{PDCA}{Plan-Do-Check-Act}


%**************************************************************
% Glossario
%**************************************************************
%\renewcommand{\glossaryname}{Glossario}

\newglossaryentry{gantt}
{
	name=\glslink{gantt}{Gantt},
	text=Gantt,
	sort=gantt,
	description={Il diagramma di Gantt è uno strumento di supporto alla gestione dei progetti che permette la rappresentazione grafica di un calendario di attività. Esso è utile al fine di pianificare, coordinare e tracciare specifiche attività in un progetto dandone una chiara illustrazione dello stato d'avanzamento.}
}

\newglossaryentry{slack}
{
	name=\glslink{slack}{Slack},
	text=slack,
	sort=slack,
	description={periodo di inattività.}
}

\newglossaryentry{singleton}
{
	name=\glslink{singleton}{Singleton},
	text=Singleton,
	sort=Singleton,
	description={è un design pattern creazionale che ha lo scopo di garantire che di una determinata classe venga creata una e una sola istanza, e di fornire un punto di accesso globale a tale istanza.}
}

\newglossaryentry{strategy}
{
	name=\glslink{strategy}{Strategy},
	text=Strategy,
	sort=Strategy,
	description={è uno dei pattern comportamentali. Viene utilizzato per definire una famiglia di algoritmi, incapsularli e renderli intercambiabili (in base ad una qualche condizione) in modo trasparente al client che ne fa uso. Il pattern è utile in quelle situazioni dove sia necessario modificare dinamicamente gli algoritmi utilizzati da un'applicazione.}
}

\newglossaryentry{daog}
{
	name=\glslink{dao}{DAO},
	text=Data Access Object,
	sort=dao,
	description={è un pattern architetturale per la gestione della persistenza: si tratta fondamentalmente di una classe con relativi metodi che rappresenta un'entità tabellare di un RDBMS, usata principalmente in applicazioni web, per stratificare e isolare l'accesso ad una tabella/record tramite query (poste all'interno dei metodi della classe) ovvero al data layer da parte della business logic creando un maggiore livello di astrazione ed una più facile manutenibilità. I metodi del DAO con le rispettive query dentro verranno così richiamati dalle classi della business logic.}
}

\newglossaryentry{repository}
{
	name=\glslink{repository}{Repository},
	text=repository,
	sort=repository,
	description={database in grado di contenere svariate tipologie di dati, corredate da relative informazioni (metadati). Offre inoltre un sistema di versionamento in grado di tener traccia delle modifiche effettuate al suo interno.}
}

\newglossaryentry{jvmg}
{
	name=\glslink{JVM}{JVM},
	text=JVM,
	sort=JVM,
	description={è la componente della piattaforma Java che esegue i programmi tradotti in bytecode dopo una prima compilazione}
}

\newglossaryentry{jsong}
{
	name=\glslink{JSON}{JSON},
	text=JSON,
	sort=JSON,
	description={è un formato adatto per lo scambio dei dati in applicazioni client-server}
}

\newglossaryentry{urig}
{
	name=\glslink{URI}{URI},
	text=URI,
	sort=URI,
	description={la locuzione Uniform Resource Identifier in informatica, si riferisce a una stringa che identifica univocamente una risorsa generica che può essere un indirizzo Web, un documento, un’immagine, un file, un servizio, un indirizzo di posta elettronica, ecc..}
}

\newglossaryentry{restg}
{
	name=\glslink{REST}{REST},
	text=REST,
	sort=REST,
	description={un tipo di architettura software per i sistemi di ipertesto distribuiti come il World Wide Web}
}

\newglossaryentry{ideg}
{
	name=\glslink{IDE}{IDE},
	text=IDE,
	sort=IDE,
	description={software che, in fase di programmazione, aiuta i programmatori nello sviluppo del codice sorgente di un programma. Spesso aiuta lo sviluppatore segnalando errori di sintassi del codice direttamente in fase di scrittura, oltre a tutta una serie di strumenti e funzionalità di supporto alla fase di sviluppo e debugging}
}

\newglossaryentry{git}
{
	name=\glslink{git}{Git},
	text=git,
	sort=git,
	description={sistema software di controllo versione}
}

\newglossaryentry{framework}
{
	name=\glslink{framework}{Framework},
	text=framework,
	sort=framework,
	description={in informatica, e specificatamente nello sviluppo software, un framework è un'architettura logica di supporto su cui un software può essere progettato e realizzato, spesso facilitandone lo sviluppo da parte del programmatore}
}

\newglossaryentry{mvcg}
{
	name=\glslink{mvc}{MVC},
	text=Model-View-Controller,
	sort=mvc,
	description={il Model-View-Controller, in informatica, è un pattern architetturale molto diffuso nello sviluppo di sistemi software, in particolare nell'ambito della programmazione orientata agli oggetti, in grado di separare la logica di presentazione dei dati dalla logica di business}
}

\newglossaryentry{cmsg}
{
	name=\glslink{cms}{CMS},
	text=Content Management System,
	sort=cms,
	description={strumento software installato su un server web studiato per facilitare la gestione dei contenuti dei siti web, svincolando l'amministratore da conoscenze tecniche di programmazione}
}

\newglossaryentry{ictg}
{
	name=\glslink{ict}{ICT},
	text=Information and Communications Technology,
	sort=ict,
	description={l'insieme dei metodi e delle tecnologie che realizzano i sistemi di trasmissione, ricezione ed elaborazione di informazioni (tecnologie digitali comprese)}
}

\newglossaryentry{apig}
{
    name=\glslink{api}{API},
    text=Application Program Interface,
    sort=api,
    description={in informatica con il termine \emph{Application Programming Interface API} (ing. interfaccia di programmazione di un'applicazione) si indica ogni insieme di procedure disponibili al programmatore, di solito raggruppate a formare un set di strumenti specifici per l'espletamento di un determinato compito all'interno di un certo programma. La finalità è ottenere un'astrazione, di solito tra l'hardware e il programmatore o tra software a basso e quello ad alto livello semplificando così il lavoro di programmazione}
}

\newglossaryentry{umlg}
{
    name=\glslink{uml}{UML},
    text=UML,
    sort=uml,
    description={in ingegneria del software \emph{UML, Unified Modeling Language} (ing. linguaggio di modellazione unificato) è un linguaggio di modellazione e specifica basato sul paradigma object-oriented. L'\emph{UML} svolge un'importantissima funzione di ``lingua franca'' nella comunità della progettazione e programmazione a oggetti. Gran parte della letteratura di settore usa tale linguaggio per descrivere soluzioni analitiche e progettuali in modo sintetico e comprensibile a un vasto pubblico}
}

 % database di termini
\makeglossaries


%**************************************************************
% Impostazioni di graphicx
%**************************************************************
\graphicspath{{immagini/}} % cartella dove sono riposte le immagini


%**************************************************************
% Impostazioni di hyperref
%**************************************************************
\hypersetup{
    %hyperfootnotes=false,
    %pdfpagelabels,
    %draft,	% = elimina tutti i link (utile per stampe in bianco e nero)
    colorlinks=true,
    linktocpage=true,
    pdfstartpage=1,
    pdfstartview=FitV,
    % decommenta la riga seguente per avere link in nero (per esempio per la stampa in bianco e nero)
    %colorlinks=false, linktocpage=false, pdfborder={0 0 0}, pdfstartpage=1, pdfstartview=FitV,
    breaklinks=true,
    pdfpagemode=UseNone,
    pageanchor=true,
    pdfpagemode=UseOutlines,
    plainpages=false,
    bookmarksnumbered,
    bookmarksopen=true,
    bookmarksopenlevel=1,
    hypertexnames=true,
    pdfhighlight=/O,
    %nesting=true,
    %frenchlinks,
    urlcolor=webbrown,
    linkcolor=RoyalBlue,
    citecolor=webgreen,
    %pagecolor=RoyalBlue,
    %urlcolor=Black, linkcolor=Black, citecolor=Black, %pagecolor=Black,
    pdftitle={\myTitle},
    pdfauthor={\textcopyright\ \myName, \myUni, \myFaculty},
    pdfsubject={},
    pdfkeywords={},
    pdfcreator={pdfLaTeX},
    pdfproducer={LaTeX}
}

%**************************************************************
% Impostazioni di itemize
%**************************************************************
%\renewcommand{\labelitemi}{$\ast$}

\renewcommand{\labelitemi}{$\bullet$}
%\renewcommand{\labelitemii}{$\cdot$}
%\renewcommand{\labelitemiii}{$\diamond$}
%\renewcommand{\labelitemiv}{$\ast$}


%**************************************************************
% Impostazioni di listings
%**************************************************************
\lstset{
    language=[LaTeX]Tex,%C++,
    keywordstyle=\color{RoyalBlue}, %\bfseries,
    basicstyle=\small\ttfamily,
    %identifierstyle=\color{NavyBlue},
    commentstyle=\color{Green}\ttfamily,
    stringstyle=\rmfamily,
    numbers=none, %left,%
    numberstyle=\scriptsize, %\tiny
    stepnumber=5,
    numbersep=8pt,
    showstringspaces=false,
    breaklines=true,
    frameround=ftff,
    frame=single
} 


%**************************************************************
% Impostazioni di xcolor
%**************************************************************
\definecolor{webgreen}{rgb}{0,.5,0}
\definecolor{webbrown}{rgb}{.6,0,0}


%**************************************************************
% Altro
%**************************************************************

\newcommand{\omissis}{[\dots\negthinspace]} % produce [...]

% eccezioni all'algoritmo di sillabazione
\hyphenation
{
    ma-cro-istru-zio-ne
    gi-ral-din
}

\newcommand{\sectionname}{sezione}
\addto\captionsitalian{\renewcommand{\figurename}{figura}
                       \renewcommand{\tablename}{tabella}}

\newcommand{\glsfirstoccur}{\ap{{[g]}}}

\newcommand{\intro}[1]{\emph{\textsf{#1}}}

%**************************************************************
% Environment per ``rischi''
%**************************************************************
\newcounter{riskcounter}                % define a counter
\setcounter{riskcounter}{0}             % set the counter to some initial value

%%%% Parameters
% #1: Title
\newenvironment{risk}[1]{
    \refstepcounter{riskcounter}        % increment counter
    \par \noindent                      % start new paragraph
    \textbf{\arabic{riskcounter}. #1}   % display the title before the 
                                        % content of the environment is displayed 
}{
    \par\medskip
}

\newcommand{\riskname}{Rischio}

\newcommand{\riskdescription}[1]{\textbf{\\Descrizione:} #1.}

\newcommand{\risksolution}[1]{\textbf{\\Soluzione:} #1.}

%**************************************************************
% Environment per ``use case''
%**************************************************************
\newcounter{usecasecounter}             % define a counter
\setcounter{usecasecounter}{0}          % set the counter to some initial value

%%%% Parameters
% #1: ID
% #2: Nome
\newenvironment{usecase}[2]{
    \renewcommand{\theusecasecounter}{\usecasename #1}  % this is where the display of 
                                                        % the counter is overwritten/modified
    \refstepcounter{usecasecounter}             % increment counter
    \vspace{10pt}
    \par \noindent                              % start new paragraph
    {\large \textbf{\usecasename #1: #2}}       % display the title before the 
                                                % content of the environment is displayed 
    \medskip
}{
    \medskip
}

\newcommand{\usecasename}{UC}

\newcommand{\usecaseactors}[1]{\textbf{\\Attori Principali:} #1. \vspace{4pt}}
\newcommand{\usecasepre}[1]{\textbf{\\Precondizioni:} #1. \vspace{4pt}}
\newcommand{\usecasedesc}[1]{\textbf{\\Descrizione:} #1. \vspace{4pt}}
\newcommand{\usecasepost}[1]{\textbf{\\Postcondizioni:} #1. \vspace{4pt}}
\newcommand{\usecasealt}[1]{\textbf{\\Scenario Alternativo:} #1. \vspace{4pt}}

%**************************************************************
% Environment per ``namespace description''
%**************************************************************

\newenvironment{namespacedesc}{
    \vspace{10pt}
    \par \noindent                              % start new paragraph
    \begin{description} 
}{
    \end{description}
    \medskip
}

\newcommand{\classdesc}[2]{\item[\textbf{#1:}] #2}